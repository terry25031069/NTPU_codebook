\documentclass[a4paper,8pt,twocolumn,oneside]{article}
\setlength{\columnsep}{5pt}                    %兩欄模式的間距
\setlength{\columnseprule}{0pt}                 %兩欄模式間格線粗細

\usepackage{amsthm}								%定義,例題
\usepackage{amssymb}
\usepackage{fontspec}							%設定字體
\usepackage{color}
\usepackage[x11names]{xcolor}
\usepackage{listings}							%顯示code用的
\usepackage{fancyhdr}							%設定頁首頁尾
\usepackage{graphicx}							%Graphic
\usepackage{enumerate}
\usepackage{titlesec}
\usepackage{amsmath}
\usepackage[CheckSingle, CJKmath]{xeCJK}
 \usepackage{CJKulem}

\usepackage{amsmath, courier, listings, fancyhdr, graphicx}
\topmargin=0pt
\headsep=5pt
\textheight=780pt
\footskip=0pt
\voffset=-40pt
\textwidth=545pt
\marginparsep=0pt
\marginparwidth=0pt
\marginparpush=0pt
\oddsidemargin=0pt
\evensidemargin=0pt
\hoffset=-42pt

\titlespacing\section{0pt}{-2pt plus 0pt minus 2pt}{-1pt plus 0pt minus 2pt}
\titlespacing\subsection{0pt}{-2pt plus 0pt minus 2pt}{-1pt plus 0pt minus 2pt}
\titlespacing\subsubsection{0pt}{-2pt plus 0pt minus 2pt}{-1pt plus 0pt minus 2pt}

%\renewcommand\listfigurename{圖目錄}
%\renewcommand\listtablename{表目錄}

%%%%%%%%%%%%%%%%%%%%%%%%%%%%%

\setmainfont{Consolas}
%\setmonofont{Ubuntu Mono}
\setmonofont{Consolas}
\setCJKmainfont{Noto Sans TC Regular}
\XeTeXlinebreaklocale "zh"						%中文自動換行
\XeTeXlinebreakskip = 0pt plus 1pt				%設定段落之間的距離
\setcounter{secnumdepth}{3}						%目錄顯示第三層

%%%%%%%%%%%%%%%%%%%%%%%%%%%%%
\makeatletter
\lst@CCPutMacro\lst@ProcessOther {"2D}{\lst@ttfamily{-{}}{-{}}}
\@empty\z@\@empty
\makeatother
\lstset{										% Code顯示
    language=C++,									% the language of the code
    basicstyle=\footnotesize\ttfamily, 					% the size of the fonts that are used for the code
    %numbers=left,									% where to put the line-numbers
    numberstyle=\footnotesize,					% the size of the fonts that are used for the line-numbers
    stepnumber=1,									% the step between two line-numbers. If it's 1, each line  will be numbered
    numbersep=5pt,									% how far the line-numbers are from the code
    backgroundcolor=\color{white},				% choose the background color. You must add \usepackage{color}
    showspaces=false,								% show spaces adding particular underscores
    showstringspaces=false,						% underline spaces within strings
    showtabs=false,								% show tabs within strings adding particular underscores
    frame=false,										% adds a frame around the code
    tabsize=2,										% sets default tabsize to 2 spaces
    captionpos=b,									% sets the caption-position to bottom
    breaklines=true,								% sets automatic line breaking
    breakatwhitespace=false,						% sets if automatic breaks should only happen at whitespace
    escapeinside={\%*}{*)},						% if you want to add a comment within your code
    morekeywords={*},								% if you want to add more keywords to the set
    keywordstyle=\bfseries\color{Blue1},
    commentstyle=\itshape\color{Red4},
    stringstyle=\itshape\color{Green4},
}

\usepackage{multicol} 

\begin{document}
\pagestyle{fancy}
\fancyfoot{}
%\fancyfoot[R]{\includegraphics[width=20pt]{ironwood.jpg}}
\fancyhead[L]{National Taipei University NTPU Kite }
\fancyhead[R]{(\today) \thepage}
\renewcommand{\headrulewidth}{0.4pt}
\renewcommand{\contentsname}{Contents}

\begin{multicols}{2}
\scriptsize
\tableofcontents
\end{multicols}
\section{Basic}

\subsection{compile}
\lstinputlisting{code/Basic/compile.txt}
\subsection{default code}
\lstinputlisting{code/Basic/default.cpp}
\subsection{debug list}
\lstinputlisting{code/Basic/bug_list.txt}

\section{Dark Code}

\subsection{IO optimization}
\lstinputlisting{code/DarkCode/IO_optimization.cpp}

\section{Geometry}

\subsection{2D point}
\lstinputlisting{code/Geometry/2Dpoint.cpp}
\subsection{Convex Hull}
\lstinputlisting{code/Geometry/ConvexHull.cpp}

\section{Flow}

\subsection{Dinic}
\lstinputlisting{code/Flow/dinic.cpp}
\subsection{min cost flow}
\lstinputlisting{code/Flow/Min_Cost_Max_Flow.cpp}

\section{Mathmatics}

\subsection{ax+by=gcd(a,b)}
\lstinputlisting{code/Math/ax+by=gcd.cpp}
\subsection{BigInt}
\lstinputlisting{code/Math/Bigint.cpp}
\subsection{GaussElimination}
\lstinputlisting{code/Math/GaussElimination.cpp}
\subsection{Inverse}
\lstinputlisting{code/Math/Inverse.cpp}
\subsection{LinearPrime}
\lstinputlisting{code/Math/LinearPrime.cpp}
\subsection{Miller Rabin}
\lstinputlisting{code/Math/Miller-Rabin.cpp}
\subsection{Pollard's rho}
\lstinputlisting{code/Math/pollardRho.cpp}
\subsection{數論基本工具}
\lstinputlisting{code/Math/number_tool.cpp}
\subsection{Mobius}
\lstinputlisting{code/Math/Mobius.cpp}
\subsection{SG}
\lstinputlisting{code/Math/Sprague-Grundy.cpp}
\subsection{Theorem}
\lstinputlisting{code/Math/theorem.cpp}


\section{Graph}

\subsection{BCC}
\lstinputlisting{code/Graph/BCC_edge.cpp}
\subsection{Prim}
\lstinputlisting{code/Graph/Prim.cpp}
\subsection{Bellman Ford}
\lstinputlisting{code/Graph/Bellman_ford.cpp}
\subsection{Kruskal}
\lstinputlisting{code/Graph/Kruskal.cpp}
\subsection{Dijkstra}
\lstinputlisting{code/Graph/Dijkstra.cpp}
\subsection{Strongly Connected Component(SCC)}
\lstinputlisting{code/Graph/Kosaraju_SCC.cpp}
\subsection{Hungarian}
\lstinputlisting{code/Graph/Matching/Hungarian.cpp}
\subsection{KM}
\lstinputlisting{code/Graph/Matching/Kuhn_Munkres.cpp}
\subsection{最小平均環}
\lstinputlisting{code/Graph/Min_mean_cycle.cpp}
\subsection{偵測負環}
\lstinputlisting{code/Graph/negative_cycle_validation.cpp}
\subsection{Tarjan}
\lstinputlisting{code/Graph/Tarjan.cpp}
\subsection{Topological Sort}
\lstinputlisting{code/Graph/Topological_Sort.cpp}

\section{Data Structure}

\subsection{2D Range Tree}
\lstinputlisting{code/DataStructure/2D_RangeTree.cpp}
\subsection{Sparse Table}
\lstinputlisting{code/DataStructure/SparseTable.h}
\subsection{Segment Tree}
\lstinputlisting{code/DataStructure/Segment_tree.cpp}
\subsection{Lazy Tag}
\lstinputlisting{code/DataStructure/Lazy_tag.cpp}



\section{String}

\subsection{KMP}
\lstinputlisting{code/String/KMP.h}
\subsection{smallest rotation}
\lstinputlisting{code/String/smallest_rotation.cpp}
\subsection{Suffix Array}
\lstinputlisting{code/String/suffix_array.cpp}
\subsection{Z-value}
\lstinputlisting{code/String/Z-value.cpp}

\section{Others}

\subsection{二分搜尋法}
\lstinputlisting{code/Other/BinarySearch.cpp}
\subsection{矩陣數定理}
\lstinputlisting{code/Other/count_spanning_tree.cpp}
\subsection{1D/1D dp 優化}
\lstinputlisting{code/Other/Dp1D1D.cpp}
\subsection{Theorm - DP optimization}
\lstinputlisting{code/Other/DP-optimization.txt}
\subsection{Stable Marriage}
\lstinputlisting{code/Other/Marriage.cpp}
\subsection{python 小抄}
\lstinputlisting{code/Other/python.py}


\section{Persistence}



\end{document}
